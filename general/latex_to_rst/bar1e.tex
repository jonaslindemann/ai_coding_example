\newpage
\index{bar1e}
\lhead[\bf bar1e]{\bf One dimensional bar element}
\rhead[\bf One dimensional bar element]{\bf bar1e}
\begin{tagx}
\tag{Purpose:}
Compute element stiffness matrix for a one dimensional bar element.
\begin{figure}[hhh]
\begin{center}
\mbox{}\bild{bar1e_1.eps}
\end{center}
\end{figure}

\vspace{-10mm}
\tag{Syntax:}
{\sf{Ke=bar1e(ex,ep)}}\\
{\sf{[Ke,fe]=bar1e(ex,ep,eq)}}

\vspace{5mm}

\tag{Description:}

{\sf bar1e} provides the element stiffness matrix {\sf Ke} for a
one dimensional bar element.
The input variables
\begin{eqnarray*}
\begin{array}{l}
{\sf ex}=[\;x_1 \;\; x_2 \;]  
\end{array}
\quad\quad
{\sf ep} = \left[\;E \;A\;\right]
\end{eqnarray*}
supply the element nodal coordinates
$x_1$ and $x_2$, the modulus of elasticity $E$,
and the cross section area $A$.

The element load vector {\sf fe} can also be computed if uniformly
distributed load is applied to the element.
The optional input variable
\begin{eqnarray*}
{\sf eq} =\left[\begin{array}{cc} q_{\bar{x}}  \end{array} \right]
\end{eqnarray*}
then contains the distributed load per unit length, $q_{\bar{x}}$.

\begin{figure}[hhh]
\begin{center}
\mbox{}\bild{bar1e_2.eps}
\end{center}
\end{figure}


\tag{Theory:}

The element stiffness matrix $\bar{{\bf K}}^e$, stored in {\sf Ke}, is computed
according to
\begin{eqnarray*}
\bar{\bf K}^e=\frac{D_{EA}}L\left[
\begin{array}{rr} 1 & -1  \\ -1 & 1  \end{array} \right]
\end{eqnarray*}
where the axial stiffness $D_{EA}$ and the length $L$ are given by
\vspace{2mm}
\begin{eqnarray*}
D_{EA}=EA;\quad L= x_2-x_1 
\end{eqnarray*}

The element load vector $\bar{\bf f}_l^e$, stored in {\sf fe}, is computed
according to%
\begin{eqnarray*}
\bar{\bf f}_l^e =\frac{q_{\bar{x}}L}{2}
\left[
\begin{array}{c}  1 \\  1 \end{array} \right]
\end{eqnarray*}

\end{tagx}